\section{TASK 1, Summary of the paper contents}

\begin{frame}
	\frametitle{Summary}
	\begin{itemize}
		\item Approximation of nonlinear dynamical systems with NNs
		\item Continuous-time models (i.e. sets of ODEs)
		\item Black-Box-Approach
		\begin{itemize}
			\item[$\Rightarrow$] No information on the approximated system
			\item[$\Rightarrow$] The \textbf{entire} system is approximated
		\end{itemize}
		\item Gray-Box-Approach
		\begin{itemize}
			\item[$\Rightarrow$] Some information on the approximated system
			\item[$\Rightarrow$] Only \textbf{unknown} parts of the system are approximated
		\end{itemize}
	\end{itemize}
\end{frame}

\begin{frame}
	\frametitle{Questions}
	\begin{enumerate}
		\item Which integration rule did they use in the paper?
		\begin{itemize}
			\item[$\Rightarrow$] Trapezodial Rule.\vspace{1mm}
			\item[$\Rightarrow$] ODE: $\dot{\overrightarrow{X}} = F(\overrightarrow{X}; \overrightarrow{p})$\vspace{1mm}
			\item[$\Rightarrow$] Rule: $\overrightarrow{X}_{n+1} = \overrightarrow{X}_n + \frac{h}{2}[F(\overrightarrow{X}_n; \overrightarrow{p}) + F(\overrightarrow{X}_{n+1}; \overrightarrow{p})]$
		\end{itemize}
		\item How did they design the neural network?
		\begin{itemize}
			\item[$\Rightarrow$] RNN due to implicit integration rule
			\item[$\Rightarrow$] Two hidden layers with six neurons
			\item[$\Rightarrow$] Sigmodal activation functions
		\end{itemize}
		\item What is the dynamic system used as an illustrative example?
		\begin{itemize}
			\item[$\Rightarrow$] Reactor with a irreversible reaction on a catalytic surface
		\end{itemize}
		\item What are the results?
		\begin{itemize}
			\item[$\Rightarrow$] Good approximations with Gray-Box-Approach
		\end{itemize}
	\end{enumerate}
\end{frame}
